本研究得出了结果,高强度线偏振激光和圆偏振激光与微尺度等离子体的相互作用可以触发相对论性磁重联。
同时,本研究特别考虑在空间等离子体测量和天文观测中需要的低$\beta$环境中发生的磁重联现象。
我们考虑了磁场能量占主导地位的情况,计算出了激光入射以后的$\beta$值,
即如图\ref{fig:plasma_beta}所示的$\beta< 1$的状态。在这种情况下,磁能在重联过程中起主导作用。

我们观察到了磁场重联过程中的一些重要现象,包括电子和离子在重联区域的不同行为,图\ref{fig:side_xy:d}所示的粒子射流过程,
以及磁场能量转移的机制。
本研究中注意到了重联过程中产生的相对论性射流以及在图\ref{corona}中画出的能谱上面得到了磁场能量转移的初步结论。
这些观察结果为我们更深入地理解重联现象的物理机制提供了重要线索和数据支持。
通过对磁场能量转移可能发生的区域进行分析,我们引入了电子扩散区和离子扩散区的概念,
并探讨了它们在磁重联过程中的作用和特征。

1. 我们利用激光入射等离子体,并使等离子体的密度在$x \approx 21$处开始下降,
这可以使激光在等离子体中激发的电子束收缩靠近。通过调整激光的种类和偏振方向,
可以获得更高效的重联方案,满足不同实验需求。

2. 我们讨论了不同方案中激发高能粒子的效率。由于重联率与重联电场正相关,
通过分析不同激光条件下的重联场变化,可以了解其对重联率的影响。

3. 粒子激发效率的比较如图\ref{fig:Time_Dependence_of_Total_Energy_Increase_Combined}所示。
可以看到,圆偏振激光在重联过程中比线偏振激光的激发效率提高了约$50\%$。

接着,我们探讨了在线偏振激光和圆偏振激光触发的磁重联过程中,产生的非热粒子能谱遵循幂律分布
,并且其指数分别为1.8639和1.9047。

圆偏振激光触发的磁重联过程产生的幂律能谱指数(1.9047)比在线偏振激光触发的更高(1.8639)。
这说明圆偏振激光引起的磁重联可能导致更多高能粒子的产生,而在线偏振激光则相对较少。同时也可得出,
在线偏振激光可能引起的磁重联过程可能与圆偏振激光相比,加速机制更为平缓。

文章的最后根据统计出来的能量计算了等离子体的$\beta$参数,并探讨了$\beta$参数在等离子体诊断中的作用,从而确定“磁场主导”的等离子体。

