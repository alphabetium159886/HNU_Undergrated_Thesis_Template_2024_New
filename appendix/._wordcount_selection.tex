
大学四年一转眼就过去了,在这段宝贵的时光里,我不仅学到了丰富的知识,更收获了许多宝贵的友情和支持。
在此,我谨向在我大学阶段给予我帮助和鼓励的各位表达最诚挚的感谢。

首先,我要特别感谢我的导师肖成卓老师。
在整个研究过程中,他的悉心指导和无私帮助。除了在理论层面给我方向以外,在实践层面上给我们提供了超算的资源让我们使用,
同时也亲自指导我们分析模拟结果,在论文的写作和修改上面给我大量的帮助和支持。使我在学术道路上不断进步。他的专业知识、耐心和鼓励是我完成这篇论文的关键。

同时,我要感谢组里的各位同学和学长姐们。在实验和研究过程中,和他们的合作与帮助为我提供了宝贵的支持和启发。
特别感谢谭尚学长在研究中的讨论与建议,在指导我入门EPOCH的时候教我大量的相关知识,
并且也在我做毕设的时候多次主动询问并帮我解决问题,使我的研究工作更加完善。

此外,我要感谢我的家人和朋友们。你们在我大学阶段给予了我无尽的支持和鼓励。
特别是我的家人们,你们的理解和支持是我前进的动力。

特别感谢我的好朋友们,我们共同度过了这段难忘的时光。和我一起组队来湖大上学的朋友,张祐、晨右、若羽、雅羚,大家都是我在大学生活学习非常重要的伙伴。
还要感谢班里的好朋友们,班长、周子诺、小翟、赖师傅,在大学四年的过程中一起学习讨论吃喝玩乐,在我情绪低谷陪我到处乱跑。
你们所有人的友谊和陪伴让我在学习生活中充满了乐趣和动力。四年的时光匆匆而过,但我们一起经历的点滴将永远铭记在心。

最后,我要感谢所有在我的大学生涯中给予我帮助和支持的人们,虽然无法在此一一列举,但你们的每一份帮助和鼓励都铭记在心。

再次感谢大家!
